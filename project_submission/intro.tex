\section{Introduction}
Physical devices connected to the internet, also known as Internet-of-things (IoT), have gained popularity in recent years. Among numerous types of IoT devices, smart speakers with voice assistants, such as Amazon Echo and Google Home, are widely seen in users' home. These devices detect and respond to voice commands. They normally have always-on microphones, which theoretically start recording after hearing a wake word and then send the voice data to a server via internet for further processing~\cite{AmazonEc68:online}.

This behavior has sparked many privacy concerns, including but not limited to what is being recorded, how the collected data is used and stored, and whether it is being protected well~\cite{lau2018alexa, fowler_2019, apthorpe2017smart, apthorpe2019keeping, apthorpe2017spying}. Much of previous work has been centered around protecting sensitive information from being leaked to adversaries~\cite{apthorpe2017smart, apthorpe2019keeping, apthorpe2017spying}. Other questions such as what is being recorded and whether Echo is eavesdropping on users or not remain unanswered. As an attempt to increase transparency, Amazon has made it possible for users to view, play and delete the voice data transmitted to its server~\cite{ford2019alexa}. However, there is no guarantee that the audio made available by Amazon is all that collected. It is still possible that Echo streams unwanted audio data that happens before or after the command to its server. To our best knowledge, no previous work has tried to mearsure whether Echo is transmitting anything other than the command.

This is where our work comes in. We designed and implemented experiments that allowed us to infer the behaviour of Echo through network traffic analysis. In our experiment setup, we connected Echo to a dedicated hotspot and used wireshark to capture all the packets sent and received by Echo. To activate Echo, an audio command was needed. We prerecorded an audio file that contained a simple command ("Alexa, where is New York City?") that was about 2.5 seconds long. In this audio file, we also prefixed and suffixed the command with irrelevant conversations. This prerecorded audio gave us both repeatibility and flexibility. First, by tuning when to start and stop, different segments of this audio could be played. Second, the same segment could be played over and over again, with the guarantee that the input stays the same. We then carried out a set of experiments with this kind of setup.

\textbf{Our first contribution: we confirmed that usually Echo only transmitted audio that happened after the wake word.} For this experiment, we used used two segments from the prerecorded audio. The first audio segment (denoted as $I_1$) was about 2.5 seconds long and contained only the Echo command "Alexa, where is New York City?". The other audio segment (denoted as $I_2$) was about 11.5 seconds long, with a 9 seconds irrelevant conversation attached right before the Echo command. We played both $I_1$ and $I_2$ alternatively overnight for over 150 times, giving us in total more than 300 samples. For each sample, after data cleaning, we calculated the total size of all outgoing traffic. We then computed the mean and standard deviation of the total size across all samples respectively for $I_1$ and $I_2$, as shown in Table 1\todo{refer to table}. We did not observe any significant difference in the total size of outgoing packets between $I_1$ and $I_2$. Our hypothesis was that if Echo was sending any audio that happened before the wake word, we should be able to observe differen total sizes of outgoing traffic, which we did not. Thus, we concluded that nothing before the wake word was streamed to the server.

That being said, we did observe a few outliers that sent significant more data then expected. Since we were unable to decrypt the traffic, we were unsure if our conclusion would still hold for these corner cases.

\textbf{Our second contribution: we confirmed that the Echo does not record any audio past the end of a command when such an end is properly detected. However if the Echo fails to detect the end of the command, it will continue to record audio.}

\textbf{Our third contribution: we identify the necessary pause detection of the end of a command, and determine that such detection is done locally on the Echo and not on the Amazon servers.}



\todo{Other experiments we did.}

%\section{Goals}

%We want to quantify how far beyond the bounds (start and end) of a particular command the smart speaker records and transmits to the cloud. We want to test how the ambient noise levels (e.g. music or conversation) surrounding the command affect the length of this extra recording.

%We also want to know how the smart speaker decides when to stop recording and transmitting to the cloud. % Does it receive a stop command from the cloud, or does it stop on its own, either through detecting the end of the command through some heuristic or a simple time-out? In particular, if the ending condition is missed for whatever reason, how long will the smart speaker record and upload to the cloud before stopping? \textcolor{red}{We would imagine that Alexa stop recording when they first find a word that should not belong to the command or they just stop recording automatically (or after a few timeouts). We can infer the algorithm that Alexa uses and let user know how to protect their privacy.}
%Finally, %\textcolor{red}{after a test, we find out that sometimes Alexa started recording by a word before the "wake up" command (i.e."Alexa") which made the command meaningless. We want to detect why this will happen and} 
%we want to know if recordings are being made and sent to the cloud without being reported to the user by the service provider (e.g. through an app or API).