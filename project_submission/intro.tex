\section{Introduction}
Physical devices connected to the internet, also known as Internet-of-things (IoT), have gained popularity in recent years. Among numerous types of IoT devices, smart speakers with voice assistants, such as Amazon Echo and Google Home, are widely seen in users' home. These devices detect and respond to voice commands. They normally have always-on microphones, which theoretically start recording after hearing a wake word and then send the voice data to a server via internet for further processing~\cite{AmazonEc68:online}.

This behavior has sparked many privacy concerns, including but not limited to what is being recorded, how the collected data is used and stored, and whether it is being protected well~\cite{lau2018alexa, fowler_2019, apthorpe2017smart, apthorpe2019keeping, apthorpe2017spying}. Much of previous work has been centered around protecting sensitive information from being leaked to adversaries~\cite{apthorpe2017smart, apthorpe2019keeping, apthorpe2017spying}. However, little work has been done to confirm the rudimentary assumption -- that Echo should only start recording after hearing the wake word. Though Amazon has made it possible for users to view, play and delete the voice data transmitted to its server~\cite{ford2019alexa}, we are not fully convinced that the voice data made available is the only data transmitted by Echo everytime. More specifically, we want to know if Echo is streaming any conversation that happened before the wake word. \todo{and what?} 

To answer all these questions, we opted for a passive mearsurement approach --- analyzing network traffic sent by Echo. We used wireshark to capture all the packets transmitted by Echo. An audio that could trigger Echo to actively send data was also needed. We prerecorded an audio file that contained an Echo command ("Alexa, where is New York City?") that was about 2 seconds long. In this audio file, we also prefixed and suffixed the Echo command with irrelevant conversations. This prerecorded audio file gave us the ability to perform controlled experiments and verify our assumptions. Different segments of this audio were played to Echo over and over again, and each time we would compute the total package size transmitted. Having done this, for each unique segment of the audio, we now had a set of total packet sizes that Echo transmitted when hearing this segment of audio. Then, statistical analysis was to performed to test our assumptions.

\textbf{Our first contribution: we confirmed that xxx.} What we did, and then what are the outsiders.

\textbf{Our Second contribution: we confirmed that xxx.} \todo{Other experiments we did.}

%\section{Goals}

%We want to quantify how far beyond the bounds (start and end) of a particular command the smart speaker records and transmits to the cloud. We want to test how the ambient noise levels (e.g. music or conversation) surrounding the command affect the length of this extra recording.

%We also want to know how the smart speaker decides when to stop recording and transmitting to the cloud. % Does it receive a stop command from the cloud, or does it stop on its own, either through detecting the end of the command through some heuristic or a simple time-out? In particular, if the ending condition is missed for whatever reason, how long will the smart speaker record and upload to the cloud before stopping? \textcolor{red}{We would imagine that Alexa stop recording when they first find a word that should not belong to the command or they just stop recording automatically (or after a few timeouts). We can infer the algorithm that Alexa uses and let user know how to protect their privacy.}
%Finally, %\textcolor{red}{after a test, we find out that sometimes Alexa started recording by a word before the "wake up" command (i.e."Alexa") which made the command meaningless. We want to detect why this will happen and} 
%we want to know if recordings are being made and sent to the cloud without being reported to the user by the service provider (e.g. through an app or API).