\section{Introduction}
Internet-of-things (IoT), which refers to phsycial devices connected to the interent, have gained its popularity in recent years. Among numerous types of IoT devices, smart speakers, such as Amazon Echo and Google Home, are widely seen in users' home. These devices would start recording after hearing a wake word (Alexa, Echo, etc.) and send the recorded voice to a server through internet for further processing. This behaviour has sparked widely privacy concerns\colorbox{red}{todo:cite}. Previous work focused heavily on preserving a user's privacy from a defender's view\colorbox{red}{todo:cite}. Little has been done to investigate privacy issues from a user's perspective. Specifically, it has been reported that Amazon would keep the recorded voice data being sent to them\colorbox{red}{todo:cite}. Up to now, we lack the insight into how much is being recorded and sent to a server by these devices. This work tries to answer this question through empirical measurement.


%This is a sample!  It does not have a lot of fancy stuff in it, but if
%you want to see a more complex sample, look at the original ACM
%templates.

%To fill up enough text to fill up 3 pages, we've included the CCS Call
%for Papers three times.  It is always a good idea to include some
%gratuitious citations to recent CCS papers~\cite{medvinsky1993netcash,
%  bellare1993random, anderson1993cryptosystems, blaze1993cryptographic}.

\input{cfp}
\input{cfp}
\input{cfp}

\section{Conclusions}

In conclusion, it is rarely a good idea to include the same section three times in a paper, or to have a conclusion that does not conclude.

\appendix

\section{Location}

Note that in the new ACM style, the Appendices come before the References.

\input{cfp}

%\begin{acks}
% TODO: For the submission, don't include acknowledgments since they would most likely deanonymize you.
%\end{acks}
