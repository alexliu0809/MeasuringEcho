\section{Introduction}
Phsycial devices connected to the interent, also known as Internet-of-things (IoT), have gained its popularity in recent years. Among numerous types of IoT devices, smart speakers with voice assitants, such as Amazon Echo and Google Home, are widely seen in users' home. These devices would detect and respond to voice commands. They normally have always-on mircophones, which theoratically would start recording after hearing a wake word and then send the collected data to a server via internet for further processing \cite{AmazonEc68:online}.

This behaviour has sparked many privacy concerns, including but not limited to what is being recorded, how the collocted data is used and stored, and whether it is being protected well \cite{lau2018alexa, fowler_2019, apthorpe2017smart, apthorpe2019keeping, apthorpe2017spying}. Much of previous work has been centered around protecting sensitive information from being leaked to adversaries \cite{apthorpe2017smart, apthorpe2019keeping, apthorpe2017spying}. As for what data is being collected, Amazon has made it possible for users to view, play and delete their voice data stored on its server \cite{ford2019alexa}. However, our intuition is that Amazon would reveal to users only part of the voice data being collected by its smart speakers. Up until now, little has been done to verify whether Amazon is being honest with us or not -- is Amazon making public exactly the audio data it collects?

Our perliminary experiments with Amazon Echo have found that the voice data made public of the same command is not consistent across different trials. This stimulates us to question and examine the behaviour of Amazon Echo and dig into what is exactly being collected by it. Previous researchers \cite{apthorpe2017spying} have successfully inferred user activies using network traffic measurement. We are hoping to get some insights into this question leveraging the same technique.