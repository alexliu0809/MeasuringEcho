\section{Introduction}
Phsycial devices connected to the interent, also known as Internet-of-things (IoT), have gained its popularity in recent years. Among numerous types of IoT devices, smart speakers with voice assitants, such as Amazon Echo and Google Home, are widely seen in users' home. These devices would detect and respond to voice commands. They normally have always-on mircophones, which theoratically would start recording after hearing a wake word and then send the voice data to a server via internet for further processing \cite{AmazonEc68:online}.

This behaviour has sparked many privacy concerns, including but not limited to what is being recorded, how the collocted data is used and stored, and whether it is being protected well \cite{lau2018alexa, fowler_2019, apthorpe2017smart, apthorpe2019keeping, apthorpe2017spying}. Much of previous work has been centered around protecting sensitive information from being leaked to adversaries \cite{apthorpe2017smart, apthorpe2019keeping, apthorpe2017spying}. As for what data is being collected, Amazon has made it possible for users to view, play and delete their voice data stored on its server \cite{ford2019alexa}. However, our intuition is that Amazon would reveal to users only part of the voice data being collected by its smart speakers. Up until now, little has been done to verify whether Amazon is being honest with us or not -- is Amazon making public exactly the audio data it collects?

Researchers from Princeton have successfully infered user activity using network traffic measurement \cite{apthorpe2017spying}, building on top of which others have been able to establish network signatures for Amazon Echo's network traffic. We plan to use the same technique for our work, and it is turns out to be working, we will be able to answer a whole set of questions.
%Our expectation of Amazon Echo is that if it is given the same audio input, it should stop recording at the same place everytime. Our perliminary experiments with Amazon Echo have found that the recorded voice of the same command is not consistent across different trials, suggesting something different. This stimulates us to question and examine the behaviour of Amazon Echo and dig into what is exactly being collected by it. We are hoping to get some insights into this question using network traffic measurement, as previous researchers \cite{apthorpe2017spying} have successfully inferred users' activies leveraging the same technique.