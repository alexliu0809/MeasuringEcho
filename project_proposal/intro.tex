\section{Introduction}
Internet-of-things (IoT), which refers to phsycial devices connected to the interent, have gained its popularity in recent years. Among numerous types of IoT devices, smart speakers, such as Amazon Echo and Google Home, are widely seen in users' home. These devices would start recording after hearing a wake word (Alexa, Echo, etc.) and send the recorded audio to a server through internet for further processing. This behaviour has sparked many privacy concerns \cite{fowler_2019, apthorpe2017smart, apthorpe2019keeping, apthorpe2017spying, acar2018peek}. Previous work \cite{apthorpe2017smart, apthorpe2019keeping, apthorpe2017spying, acar2018peek} focused heavily on preserving a user's privacy from a defender's view. Little has been done to investigate privacy issues from a user's perspective. Specifically, it has been reported that Amazon would keep the recorded voice data being sent to them \cite{kelly_statt_2019, osborne_2019}. Up to now, we lack the insight into how much is being recorded and sent to a server by these devices. This work tries to answer this question through empirical network traffic measurement.